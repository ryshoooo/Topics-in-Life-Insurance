\documentclass[a4paper]{article}
\usepackage[utf8]{inputenc}
\usepackage{latexsym}
\usepackage{a4wide}
\usepackage{amsmath, amssymb}
\usepackage{graphicx}
\usepackage{epstopdf}
\usepackage{pstricks-add}
\usepackage{pst-plot}
\usepackage{pst-math}
\usepackage{caption}
\usepackage{mathrsfs}
\usepackage{subcaption}
\usepackage{kmath,kerkis}
\renewcommand{\tan}{\mathrm{tg}}
\newcommand{\R}{\mathbb{R}}
\newcommand{\Z}{\mathbb{Z}}
\newcommand{\Q}{\mathbb{Q}}
\newcommand{\C}{\mathbb{C}}
\newcommand{\E}{\mathbb{E}}
\newcommand{\I}{\mathbb{I}}
\newcommand{\N}{\mathbb{N}}
\newcommand{\hVec}[2]{\left(#1,#2\right)^T}
\newcommand{\vVec}[2]{\left(\begin{array}{c}
#1\\ #2
\end{array}\right)}
\newcommand{\prim}[3]{\left[#1\right]^{#2}_{#3}}
\newcommand{\id}{\mathrm{id}}
\frenchspacing
\pagestyle{plain}
\bibliographystyle{plainnat}
\begin{document}
\pagenumbering{gobble}
\begin{center}

%%% Titulní strana
%%% Tato stránka se nepřekládá do slovenštiny!!

{\large University of Copenhagen}

\medskip
{\large Faculty of SCIENCE}

\vfill
{\bfseries\Large Continuous Time Finance 2 - Topics in Life Insurance}

\vfill
\centerline{\mbox{\includegraphics[width=60mm]{LOGO.jpg}}}

\vfill
\vspace{5mm}

{\LARGE Richard Németh}\\
{\LARGE Felipe Aguirre}\\

\vspace{15mm}

%%% Název práce  v češtině přesně podle zadání
{\LARGE\bfseries Compulsory Exercise 1}

\vfill

%%% Název katedry nebo ústavu, kde byla práce oficiálně zadána
%%% (dle Organizační struktury MFF UK) 
%%% viz http://www.mff.cuni.cz/toUTF8.cs/fakulta/struktura/sekcem.htm
Department of Mathematical Sciences
% Katedra algebry
% Katedra didaktiky matematiky
% Katedra matematické analýzy
% Katedra numerické matematiky
% Katedra pravděpodobnosti a~matematické statistiky
% Matematický ústav UK


\vfill

\begin{tabular}{rl}
Course responsible: & Rolf Poulsen \\   %% Jméno a příjmení s~tituly 
\noalign{\vspace{2mm}}
Study programmes: & MSc Programme in Actuarial Mathematics
%Studijní obor: & Obecná matematika\\
%Studijní obor: & Obecná matematika\\
%Studijní obor: & Finanční matematika\\
%Studijní obor: & Matematické metody informační bezpečnosti\\
\end{tabular}

\vfill

% Zde doplňte rok
Copenhagen 2016
\newpage

\end{center}
\newpage
\pagenumbering{arabic}

\setlength\parindent{0pt}

{\Large{\textbf{The Stop-Loss Paradox}}}\\
\vspace{0.2cm}

The questions in this exercise refers to the paper Carr \& Jarrow (1990), ``The Stop-Loss Start-Gain Paradox and Option Valuation: A New Decomposition into Intrinsic and Time Value'', Review of Financial Studies, vol. 3(3), pp. 469-492, which you will find on the course homepage. \\

What do terms ``intrinsic value'' and ``time value'' in the title mean? \\
\\
\textbf{Answer.} From decomposition (equation (15) in the paper):
$$
c(0)=\max[0,S(0)-XP(0)]+P(0)\cdot \tilde{\mathbf{E}}_0\Lambda_T(X)
$$
the term $\max[0,S(0)-XP(0)]$ is called intrinsic value (value of payoff of call option at time zero) and the second term is called time value (as expected local time's present value).\\
\\

What is the stop-loss start gain strategy (as defined around equation (3) in the paper)? Why is it called that? Why does it seem to contradict usual Black-Scholes replication and pricing?\\
\\
\textbf{Answer.} The SLST strategy is portfolio
\begin{align*}
h_0(t) &= -\textbf{1}_{(S(t)>XP(t))}X\\
h_1(t) &= \textbf{1}_{(S(t)>XP(t))},
\end{align*}
where $P(t)=e^{-r(T-t)}.$ The strategy basically says to borrow $X$ units from bank account and buy one stock in case we are at-the-money, otherwise do nothing. Being at-the-money basically means that we can only gain from the stock, thus the title of this strategy. The strategy seems to be self-financing, the value of the portfolio is non-negative and with positive probability can be positive, hence this strategy could be an arbitrage possibility in non-arbitrage market. Thus the big paradox.\\

\vspace{0.2cm}
Design and execute a simulation experiment that analyzes the hedge performance of the stop-loss start-gain strategy for a call-option. Analyze the results. \\

Why is the stop-loss start-gain strategy \textit{not} a paradox; ``where does the rabbit go into the hat''? (You can argue based on insights from your simulation experiment and/or theoretically. Proposition 1 the paper deals with this matter. It uses the Ito-Tanaka(-Meyer) formula, which is beyond our sphere of knowledge -- as of now; Antoine will remedy that. However, you should be able to see what goes wrong when trying to prove the self-financing condition with the standard Ito formula.)

\vspace{0.2cm}
Computation:

\vspace{0.4cm}
{\Large{\textbf{A Reflection Theorem}}}\\
\vspace{0.2cm}
In this exercise we look at the standard no-dividends Black-Scholes model We put

$$p = 1-\frac{2r}{\sigma^2}$$

and consider a simple claim with a pay-off at time $T$ specified by a pay-off function $g$. The arbitrage-free time-$t$ value is

$$\pi^{g}\left(t\right)=e^{-r\left(T-t\right)}\mathbf{E}^{\mathbb{Q}}_{t}\left(g\left(S\left(T\right)\right)\right)=e^{-r\left(T-t\right)}f\left(S\left(T\right),t\right)$$

where of course $f\left(S\left(t\right),t\right)=\mathbf{E}^{\mathbb{Q}}_{t}\left(g\left(S\left(T\right)\right)\right)$. Let $H > 0$ be a constant and define a new function $\hat{g}$ by

$$\hat{g}\left(x\right)=\left(x/H\right)^{p}g\left(H^{2}/x\right)$$

We call a simple claim with this pay-off function $g$'s \textit{reflected claim} (don't worry why).
Define the process $Z$ by

$$Z\left(t\right) = \left(\frac{S\left(t\right)}{H}\right)^{p}$$

Show that $Z\left(t\right)/Z\left(0\right)$ is a positive, mean-1 $\mathbb{Q}-martingale$.\\
\\
\textbf{Answer.} The dynamics of $Z(t)/Z(0)$ under $\mathbb{Q}$:
\begin{align*}
\text{d}\frac{Z(t)}{Z(0)} &= \text{d}\left(\frac{S(t)}{S(0)}\right)^p=\frac{1}{S(0)^p}\left(pS(t)^{p-1}\text{d}S(t)+\frac{p(p-1)}{2}S(t)^{p-2}(\text{d}S(t))^2\right)\\
&=\frac{S(t)^p}{S(0)^p}\left(pr+\frac{1}{2}p(p-1)\sigma^2\right)\text{d}t+[\ldots]\text{d}W(t).
\end{align*}
But
$$
pr+\frac{1}{2}p(p-1)\sigma^2=r-\frac{2r^2}{\sigma^2}-\frac{1}{2}\left(1-\frac{2r}{\sigma^2}\right)\frac{2r}{\sigma^2}\sigma^2=0,
$$
hence the $\text{d}t$ term vanishes, so $Z(t)/Z(0)$ is martingale. Now using the martingale property:
$$
\mathbf{E}^{\mathbb{Q}}\left[\frac{Z(t)}{Z(0)}\right]=\left.\mathbf{E}^{\mathbb{Q}}\left[\frac{Z(t)}{Z(0)}\right|\mathcal{F}_0\right]=\frac{Z(0)}{Z(0)}=1
$$
and since $S(t)>0$ then also $Z(t)>0$.\\
\\
This means that

$$\frac{d\mathbb{Q}^{Z}}{d\mathbb{Q}}=\frac{Z\left(T\right)}{Z\left(0\right)}$$

defines a probability measure $\mathbb{Q}^{Z}\sim \mathbb{Q}$. Show that

$$\pi^{\hat{g}}\left(t\right)=e^{-r\left(T-t\right)}\left(\frac{S\left(t\right)}{H}\right)^{p}\mathbf{E}^{\mathbb{Q}^{Z}}_{t}\left(g\left(\frac{H^{2}}{S\left(T\right)}\right)\right)$$
\textbf{Answer.}
Put $Y\left(t\right) = H^{2}/S\left(t\right)$. Show that

$$dY\left(t\right)=rY\left(t\right)dt+\sigma Y\left(t\right)\left(-dW^{\mathbb{Q}^{Z}}\left(t\right)\right)$$

where $W^{\mathbb{Q}^{Z}}$ is a $\mathbb{Q}^{Z}$-Brownian motion.

This means the distribution of $Y$ under $\mathbb{Q}^{Z}$ is the same as the distribution of $S$ under $\mathbb{Q}$.

Use this to argue that

$$\pi^{\hat{g}}\left(t\right)=e^{-r\left(T-t\right)}\left(\frac{S\left(t\right)}{H}\right)^{p}f\left(\frac{H^{2}}{S\left(t\right)},t\right)$$

This result linking the original claim and its reflected version has applications in pricing and hedging of barrier options, see Poulsen (2006) ``Barrier Options and Their Static Hedges: Simple Derivations and Extensions'', Quantitative Finance, vol. 6(4), pp 327-335.

\vspace{0.2cm}
Computation:


\vspace{0.4cm}
{\Large{\textbf{Options on Coupon-Bearing Bonds}}}\\
\vspace{0.2cm}

The trick in the exercise was first used in Jamshidian (1989), ``An exact bond option pricing formula'', Journal of Finance, vol. 44(1), pp 205-209. Also in this exercise we consider the Vasicek model, and let $A$ and $B$ denote the functions such that $P\left(t,T\right)=exp\left(A\left(t,T\right)-B\left(t,T\right)r\left(t\right)\right)$. We now look at a \textit{coupon bond} that makes deterministic positive payments $\alpha_1, \cdots, \alpha_N$ at dates $T_1, \cdots, T_N$. Clearly the price of this coupon bond is

$$\pi^{C}\left(t\right)=\sum_{i|T_i >t}\alpha_i P\left(t, T_i\right)$$

(It is strict inequality, ``>'', to keep in line with prices always being ex-dividend.) The last ingredient we need is a (positive) strike-$K$, expiry-$T$ European call-option on the coupon bond.

Show that there exists a unique $r^{\ast} \in\mathbb{R}$ such that $\pi^{C}\left(T\right)\geq K$ if and only if $r\left(T\right)\leq r^{\ast}$.

Define the \textit{adjusted strikes} via $K_i = exp\left(A\left(T, T_i\right)-B\left(T, T_i\right)r^{\ast}\right)$

Show that the pay-off of the call can be written as 

$$\left(\pi^{C}\left(T\right)-K\right)^{+}=\sum_{i|T_i > T}\alpha_i\left(P\left(T,T_i\right)-K_i\right)^{+}$$

and explain how (given results known from for instance Bjork) this leads to a closed-form (up to knowledge of $r^{\ast}$) expression for the price of the call on the coupon bond.\\

Suppose the current short rate is $r_0=0.02$ and that the (Q-) parameters (in our ``drift $=k\left(\theta^{Q}-x\right)$''- notation) of the Vasicek model are $\theta^{Q}=0.05$, $k=0.1$, and $\sigma=0.015$. Consider a option with $K = 4.5$, $T = 1$, $N = 5$, $T_i = i+1$, and $\alpha_i = 1$ for all $i$, and let other parameters be as in the previous exercise. Calculate the time-0 price of the call. (Numbers, please; this will involve solving an equation numerically.)

\vspace{0.2cm}
Computation:

\vspace{0.4cm}
{\Large{\textbf{A Two-Factor Interest Rate Model}}}\\
\vspace{0.2cm}

In this exercise we look at what is often called the two-factor Hull-White model.


Suppose the short rate has the form $r(t) = X_1\left(t\right)+X_2\left(t\right)$, where the X's follow Vasicek-type (also called Ornstein-Uhlenbeck) processes, i.e. $dX_{i}\left(t\right) = k_i\left(\theta_i X_i\left(t\right)\right)dt + \sigma_i dW_{i}\left(t\right)$, and $W_{1}$ and $W_{2}$ are independent Q-Brownian motions.


Explain briefly how results from Bjork makes it easy to price but zero-coupon bonds and call and put options on these. Does the Jamshidian's trick from the previous exercise for pricing coupon-bearing bonds work in this model?

What are correlations between changes in zero-coupon yields over short time horizons? What are these correlations in the standard Vasicek model?


The formulation above of the model does not look too ``sexy''. To fix that, show that we can write it as

$$dr\left(t\right)=\tilde{k}\left(\tilde{\theta}\left(t\right)r\left(t\right)\right)dt+\tilde{\sigma}d\tilde{W}\left(t\right)$$

where $\tilde{\theta}$ is an Ornstein-Uhlenbeck process. (So were clear: Your job is to write down explicitly what the things with $\tilde{.}$'s are.) This is known as rotating the model. You could ``sell'' $\tilde{\theta}$ as a ``target rate'' controlled by a central bank (depending on the general state of the economy), or give the model a ``stochastic market price of risk'' interpretation.

\vspace{0.2cm}
Computation:


\end{document}